% !TeX root = ./TMA03.tex
From equation \eqref{eq:1.4}
\[
	\deriv{x}{s} = 2\lr{c-\lambda y} - \gamma \deriv{y}{s}
\]
and from equation \eqref{eq:1.5}
\[
	\deriv{y}{s} = 2\lr{d + \lambda y} - \gamma \deriv{x}{s}
\]
Substituting into equation \eqref{eq:1.4} for $\dd y/ \dd s$ gives,
\begin{align*}
	{}&\deriv{x}{s} + \gamma \lr{2\lr{d + \lambda x} - \gamma \deriv{x}{s}} = 2\lr{c-\lambda y},\\
	{}&\deriv{x}{s} + 2 \gamma \lr{d + \lambda x} - \gamma^2 \deriv{x}{s} = 2\lr{c-\lambda y},\\
	{}&\deriv{x}{s}\lr{1 - \gamma^2} + 2 \gamma \lr{d + \lambda x}  = 2\lr{c-\lambda y},\\
	{}&\deriv{x}{s}\lr{1 - \gamma^2} = 2\lr{c-\lambda y} - 2 \gamma \lr{d + \lambda x},\\
	{}&\deriv{x}{s}\lr{1 - \gamma^2} = 2c-2\lambda y - 2 \gamma d  - 2\gamma\lambda x,\\
	{}&\deriv{x}{s}\lr{1 - \gamma^2} = -2\lambda\lr{y + \gamma x} + 2\lr{c - \gamma d},\\
	{}&x^\prime\lr{1 - \gamma^2} = -X+ D = D - X.
\end{align*}
Therefore, \marginnote{Where, $D=2\lr{c-\gamma d}$, $X=2\lambda(x-y)$ and $x^\prime = \deriv{y}{s}$.}[0.6cm]
\begin{equation}
\label{eq:1.10}
	\boxed{(1-\gamma^2)x^\prime = D - X,}
\end{equation}
as required.

Equation~\eqref{eq:1.5} is
\[
	\gamma\deriv{x}{s} + \deriv{y}{s} = 2\lr{d + \lambda x}.
\]
Substituting into equation~\eqref{eq:1.5} for $\dd x/\dd s$ from equation~\eqref{eq:1.4} gives,
\begin{align*}
	{}& \gamma \lr{2\lr{c-\lambda y} - \gamma\deriv{y}{s}} + \deriv{y}{s} = 2\lr{d + \lambda x},\\
	{}& 2\gamma\lr{c-\lambda y} - \gamma^2\deriv{y}{s} + \deriv{y}{s} = 2\lr{d + \lambda x},\\
	{}& \deriv{y}{s}\lr{1-\gamma^2} = 2\lr{d + \lambda x} - 2\gamma\lr{c-\lambda y},\\
	{}& \deriv{y}{s}\lr{1-\gamma^2} = 2d + 2\lambda x - 2\gamma c + 2\gamma\lambda y,\\
	{}& \deriv{y}{s}\lr{1-\gamma^2} = 2\lr{d - \gamma c} + 2\lambda\lr{x+\gamma y}.\\
\end{align*}
\marginnote{Where, $C=2\lr{d - \gamma c}$, $Y=2\lambda\lr{x+\gamma y}$ and $y^\prime=\deriv{y}{s}$}[0.6cm]
\begin{equation}
\label{eq:1.11}
\therefore \qquad\boxed{\lr{1-\gamma^2}y^\prime = C + Y.}
\end{equation}
as required.

Substituting for $x^\prime$ and $y^\prime$ from equations~\eqref{eq:1.10} and \eqref{eq:1.11}, respectively, into equation~\eqref{eq:1.9}
\[
	\frac{\lr{D-X}^2}{\lr{1-\gamma^2}^2} +2\gamma\frac{\lr{D-X}}{\lr{1-\gamma^2}}\frac{\lr{C+Y}}{\lr{1-\gamma^2}} + \frac{\lr{C+Y}^2}{\lr{1-\gamma^2}^2} = 1
\]
Expanding the above and cross multiplying by $\lr{1-\gamma^2}^2$ gives,
\begin{align*}
{}&D^2 -2DX + X^2 + 2\gamma\lr{D-X}\lr{C+Y} + C^2 + 2CY + Y^2 = \lr{1-\gamma^2}^2,\\
{}&D^2 -2DX + X^2 + \lr{2\gamma D -2\gamma X}\lr{C+Y} + C^2 + 2CY + Y^2 = \lr{1-\gamma^2}^2,\\
{}&D^2 -2DX + X^2 + 2\gamma CD + 2\gamma DY  -2\gamma CX -2\gamma XY + C^2 + 2CY + Y^2 = \lr{1-\gamma^2}^2,\\
{}&X^2+Y^2-2\gamma XY + C^2 + D^2 + 2\gamma CD -2DX+2\gamma DY - 2\gamma CX +2CY = \lr{1-\gamma^2}^2,\\
\end{align*}
Collecting terms in just $X$ and just $Y$ gives,
\begin{equation}
\label{eq:1.12}
X^2+Y^2-2\gamma XY +C^2+D^2+2\gamma CD - 2X\lr{D+\gamma C} + 2Y\lr{C+\gamma D} = \lr{1-\gamma^2}^2.
\end{equation}
Now, from above (see margin notes)
\begin{align*}
\lr{D + \gamma C} =& 2\lr{c-\gamma d} + \gamma\lr{2\lr{d-\gamma c}},\\
=& 2c - \cancel{2\gamma d} +  \cancel{2\gamma d} -2\gamma^2c,\\
=& 2c\lr{1-\gamma^2}.
\end{align*}
\begin{align*}
\lr{C + \gamma D} =& 2\lr{d-\gamma c} + \gamma\lr{2\lr{c-\gamma d}},\\
=& 2d - \cancel{2\gamma c} +  \cancel{2\gamma c} -2\gamma^2d,\\
=& 2d\lr{1-\gamma^2}.
\end{align*}
So,
\[
	D + \gamma C = 2c\lr{1-\gamma^2}\quad\text{and}
\]
\[
	C + \gamma D = 2d\lr{1-\gamma^2}.
\]
Substituting these expressions into equation~\eqref{eq:1.12} gives,
\[
X^2+Y^2-2\gamma XY +C^2+D^2+2\gamma CD - 4c\lr{1-\gamma^2}X + 4d\lr{1-\gamma^2}Y = \lr{1-\gamma^2}^2,
\]
as required.

The boundary conditions are
\[
	x(0)=y(0)=0,\quad x(1)=R>0,\quad y(1)=0.
\]
Now, applying these boundary conditions to equation\eqref{eq:1.12} and from the margin notes above we have:
\[
	X(t) = 2\lambda\lr{\gamma x(t) + y(t)}.
\]
\[
	Y(t) = 2\lambda\lr{x(t) + \gamma y(t)}.
\]
At $t=0$,
\[
	X(0) = 2\lambda\lr{\gamma x(0) + y(0)} = 0.
\]
\[
	Y(0) = 2\lambda\lr{x(0) + \gamma y(0)} = 0.
\]
Thus, equation~\eqref{eq:1.12} at $t=0$ reduces to 
\[
	C^2 + D^2 + 2\gamma C D = \lr{1-\gamma^2}^2.
\]
At $t=1$,
\[
	X(1) = 2\lambda\lr{\gamma x(1) + y(1)} = 2\lambda\gamma R.
\]
\[
	Y(1) = 2\lambda\lr{x(1) + \gamma y(1)} = 2\lambda R.
\]
Thus, when applied to equation~\eqref{eq:1.12},
\begin{align*}
	4\lambda^2\gamma^2R^2 +& 4\lambda^2 R^2 -2\gamma\lr{2\lambda\gamma R}\lr{2\lambda R}-4c\lr{1-\gamma^2}\lr{2\lambda\gamma R}\\
	+& 4d\lr{1-\gamma^2}\lr{2\lambda R}+C^2+D^2 + 2\gamma CD = \lr{1-\gamma^2}^2.\\\\
	4\lambda^2\gamma^2R^2 +& 4\lambda^2 R^2 - 8\lambda^2\gamma^2R^2 - 8c\lr{1-\gamma^2}\lambda\gamma R\\
	+& 8d\lr{1-\gamma^2}\lambda R + C^2+D^2 + 2\gamma CD = \lr{1-\gamma^2}^2.
\end{align*}
Note from above that $C^2+D^2+2\gamma CD = \lr{1-\gamma^2}^2$, thus
\begin{align*}
	4\lambda^2\gamma^2R^2 +& 4\lambda^2 R^2 - 8\lambda^2\gamma^2R^2 - 8c\lr{1-\gamma^2}\lambda\gamma R
	+ 8d\lr{1-\gamma^2}\lambda R = 0,
\end{align*}
which, after dividing through by $4\lambda R$, gives\marginnote{$\lambda,R\ne0$ and $\gamma^2 \ne 1$}[1 cm]
\[
	\lambda\gamma^2R + \lambda R - 2\lambda\gamma^2R - 2c\lr{1-\gamma^2}\gamma
	+ 2d\lr{1-\gamma^2} = 0,
\]
\[
	\lambda R - \lambda\gamma^2R - 2c\lr{1-\gamma^2}\gamma
	+ 2d\lr{1-\gamma^2} = 0,
\]
\[
	\lambda R\lr{1-\gamma^2} - 2c\lr{1-\gamma^2}\gamma
	+ 2d\lr{1-\gamma^2} = 0,
\]
and dividing through by $\lr{1-\gamma^2}$, gives
\[
	\lambda R - 2c\gamma+ 2d = 0.
\]
Simplifying,
\[
	\lambda R - 2\lr{c\gamma+ d} = 0,
\]
and noting that the quantity $2\lr{c\gamma+ d}$ is equal to $C$, then
\[
	\lambda R + C = 0.
\]
Finally, 
\begin{equation}
\label{eq:1.13}
	\boxed{C=-\lambda R,\quad R > 0,}
\end{equation}
as required.

Recall that $C^2+D^2+2\gamma CD = \lr{1-\gamma^2}^2$ and $C=-\lambda R,\quad R > 0$. 

Thus,
\[
	\lambda^2 R^2 + D^2 + 2\gamma\lr{-\lambda R}D = \lr{1-\gamma^2}^2,
\]
\[
	D^2 - 2\lambda\gamma RD + \lambda^2R^2 - \lr{1-\gamma^2}^2 = 0.
\]
Solving for $D$ in the above quadratic,
\begin{align*}
	D =& \frac{ 
				2\lambda\gamma R \pm \sqrt[•]{4\lambda^2\gamma^2 R^2 - 4 \lr{\lambda^2R^2 - \lr{1-\gamma^2}^2}}
	    }
	    {
	    			2
	    },\\
	D =& \lambda\gamma R \pm \sqrt[•]{\lambda^2\gamma^2 R^2 -\lambda^2R^2 + \lr{1-\gamma^2}^2},\\
	D =& \lambda\gamma R \pm \sqrt[•]{\lambda^2R^2(\gamma^2 -1) + \lr{1-\gamma^2}^2},\\
	D =& \lambda\gamma R \pm \sqrt[•]{-\lambda^2R^2(1-\gamma^2) + \lr{1-\gamma^2}^2}
\end{align*}
It is seen that $D$ will always have two real roots if $\gamma^2 > 1$, as in the expression below
\[
	D = \lambda\gamma R \pm \sqrt[•]{-\lambda^2R^2(1-\gamma^2) + \lr{1-\gamma^2}^2}
\]
$\lr{1-\gamma^2}^2 > 0$ and $\lr{1-\gamma^2} < 0$ and consequently $-\lambda^2R^2(1-\gamma^2)> 0$ making the entire quantity inside the radical positive.

Also, if $\gamma^2 < 1$ the $\lr{1-\gamma^2} > 0$ with $-\lambda^2R^2\lr{1-\gamma^2} < 0$ and the quantity $\lr{1-\gamma^2}^2 - \lambda^2R^2\lr{1-\gamma^2}>0$ provided that $\lambda^2R^2\lr{1-\gamma^2} < \lr{1-\gamma^2}^2$, that is if $\lambda R < \sqrt{\lr{1-\gamma^2}}$, there will be two real roots of $D$.

Conversely, if $\lr{1-\gamma^2}^2 - \lambda^2R^2\lr{1-\gamma^2}<0$ then $\lambda^2R^2\lr{1-\gamma^2} > \lr{1-\gamma^2}^2$, that is $\lambda R > \sqrt{\lr{1-\gamma^2}}$, there will not be any real solutions of $D$.