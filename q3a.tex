% !TeX root = ./TMA03.tex
Heron’s formula for the area $A$ of a triangle with sides of length $a, b, c > 0$ is given by
\[
	A = \sqrt{s(s-a)(s-b)(s-c)},\quad\text{where }s=\frac{a+b+c}{2}.
\]
The method of Lagrange multipliers will be used to show that for a triangle with a fixed perimeter $P = a+b+c = 2s$, then its area is maximised when the triangle is an equilateral, as follows.

For this problem $f(a,b,c) = \sqrt{s(s-a)(s-b)(s-c)}$,\marginnote{$s=\dfrac{a+b+c}{2}$}[2cm] and $g(a,b,c) = a + b + c -2s$, so that the
auxiliary function is
\begin{align*}
	\overline{F}(a,b,c,\lambda) =& f(a,b,c) - \lambda g(a,b,c),\\
	 =& \sqrt{s(s-a)(s-b)(s-c)} -\lambda\lr{a + b + c - 2s},
\end{align*}
with each of $a, b$ and $c > 0$. The following three equations relating $a, b, c$ and $\lambda$ are required:
\begin{align*}
	\pderiv{\overline{F}}{a} = 0,\quad
	\pderiv{\overline{F}}{b} = 0\text{ and }
	\pderiv{\overline{F}}{c} = 0.
\end{align*}
So,
\begin{align*}
	\pderiv{\overline{F}}{a}=& \frac{-s(s-b)(s-c)}{2\sqrt{s(s-a)(s-b)(s-c)}} - \lambda = 0,\\
	\pderiv{\overline{F}}{b}=& \frac{-s(s-a)(s-c)}{2\sqrt{s(s-a)(s-b)(s-c)}} - \lambda = 0,\\
	\pderiv{\overline{F}}{c}=& \frac{-s(s-a)(s-b)}{2\sqrt{s(s-a)(s-b)(s-c)}} - \lambda = 0.
\end{align*}
So,
\begin{align*}
	s(s-b)(s-c) =& -2\lambda A,\\
 	s(s-a)(s-c) =& -2\lambda A,\\
 	s(s-a)(s-b) =& -2\lambda A \text{ and }\\
 	a + b + c - 2s =& 0.
\end{align*}
Thus,
\[
	s(s-b)(s-c) = s(s-a)(s-c) = s(s-a)(s-b) \text{ and } s\ne 0, a, b, c.
\]
Consequently, $a = b = c$.