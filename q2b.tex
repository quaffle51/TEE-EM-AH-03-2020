% !TeX root = ./TMA03.tex%
Using the equations found in part~(a) it can be shown that $v$ is given by the real solution(s) of
$f (v) = 0$, where
\[f(v) = v^2 - 2v(1 + \sinh v) \cosh v + 1 + 2 \sinh v.\] as follows:

Recall from part~(a) that,
\begin{equation}
\label{eq:2.2}
	v = \cosh v + B \sinh v,
\end{equation}
\begin{equation}
\label{eq:2.3}
	B^2 - 1 = 2\sinh v + 2 B\cosh v,
\end{equation}
From \eqref{eq:2.2}
\begin{equation}
\label{eq:2.4}
	B = \frac{v-\cosh v}{\sinh v}.
\end{equation}
Substituting \eqref{eq:2.4} into \eqref{eq:2.3} gives,
\[
	\lr{\frac{v-\cosh v}{\sinh v}}^2 - 1 = 2\sinh v + 2\lr{\frac{v-\cosh v}{\sinh v}}\cosh v,
\]
\begin{align*}
	\frac{v^2 -2v \cosh v + \lr{\cosh^2 v - 2\sinh^2 v}}{\sinh^2 v} =\\ 
	2\sinh v + 2\lr{\frac{v-\cosh v}{\sinh v}}\cosh v,
\end{align*}
\begin{align*}
	\frac{v^2 -2v \cosh v + 1}{\sinh^2 v} =
	\frac{2\sinh^2 v + 2v\cosh v - 2\cosh^2v}{\sinh v},
\end{align*}
\begin{align*}
	\frac{v^2 -2v \cosh v + 1}{\sinh v} = -2\lr{\cosh^2v -\sinh^2v} + 2v\cosh v,
\end{align*}
\begin{align*}
	\frac{v^2 -2v \cosh v + 1}{\sinh v} = -2 + 2v\cosh v,
\end{align*}
\begin{align*}
	{}&v^2 -2v \cosh v + 1 = 2\sinh v\lr{v\cosh v - 1},\\
	{}&v^2 = 2v \cosh v - 1 + 2\sinh v\lr{v\cosh v - 1},\\
	{}&v^2 = 2v\sinh v \cosh v - 2\sinh v + 2v\cosh v -1,\\
	{}&v^2 = 2v\cosh v\lr{\sinh v + 1} - 2\sinh v -1,
\end{align*}
\begin{equation}
\label{eq:2.5}
	\boxed{\therefore f(v)=v^2 - 2v\lr{1+\sinh v}\cosh v + 1 + 2\sinh v = 0.}
\end{equation}


	