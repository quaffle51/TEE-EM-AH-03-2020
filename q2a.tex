% !TeX root = ./TMA03.tex
The integrand is 
\[
	F = y^{\prime2} + y^2,
\]
and the \el is given by
\[
	\EL,
\]
where
\[
	\pderiv{F}{y^\prime} = 2y^\prime, \quad \deriv{}{x}\lr{\pderiv{F}{y^\prime}}=\deriv{}{x}\lr{2y^\prime} = 2y^\dprime\quad\text{and}\quad \pderiv{F}{y}=2y.
\]
Thus, the \el can be written as
\[
	2y^\dprime - 2y = 0\quad\text{or}\quad \deriv{^2y}{x^2} - y = 0.
\]
The auxiliary equation is $\lambda^2 - 1 = 0$ and has general solution
\[
	y = A \cosh\lr{x} +  B \sinh\lr{x}.
\]
When $x=0,\quad y(0)=1$, so
\[
	1 = A \cosh\lr{0} +  B \sinh\lr{0},\text{ therefore } A=1.
\]
When $x=v,\quad y(v)=v$, so
\[
	v = \cosh\lr{v} +  B \sinh\lr{v},\text{ therefore } B=\frac{v-\cosh\lr{v}}{\sinh\lr{v}}.
\]

Now, using the \textbf{Transversality Condition} in which the end of the curve can be expressed in the form $y=g(x)$ and also noting that $y\lr{v}=v$, then:\marginnote{HB p.25}[-1cm]
\[
	F + \lr{g^\prime(x) - y^\prime(x)}F_{y^\prime} = 0 \text{ at } x=v,
\]
Now, $g(x) = x$ so $g^\prime(x) = 1$ and from above $F = y^{\prime2} + y^2$, hence
\[
	\lr{y^{\prime2} + y^2 } + \lr{1 - y^\prime}2y^\prime = 0,
\]
\[
	y^{\prime2} + y^2 + 2y^\prime - 2y^{\prime 2} = 0,
\]
\begin{equation}
\label{eq:2.1}
	-y^{\prime2} + 2y^\prime + y^2 = 0  \text{ at } x=v.
\end{equation}
From above $y(x) = \cosh x + B\sinh x$ and therefore $y^\prime(x) = \sinh x + B \cosh x$

Substituting these expressions into \eqref{eq:2.1} at $x=v$ gives,
\[
	-\lr{\sinh v + B \cosh v}^2 + 2\lr{\sinh v + B \cosh v} + \lr{\cosh v + B\sinh v}^2 = 0
\]
which simplifies to
\[
	-\sinh^2 v -  B^2 \cosh^2 v + 2\sinh v + 2B\cosh v + \cosh^2 v + B^2\sinh^2 v = 0
\]
Note: $\cosh^2 v - \sinh^2 v = 1$, so
\[
	1 - B^2 + 2\sinh v + 2B\cosh v = 0,
\]
\[
	\boxed{\text{therefore, } B^2 - 1 = 2\sinh v + 2 B\cosh v}
\]
