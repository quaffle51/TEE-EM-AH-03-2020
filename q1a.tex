% !TeX root = ./TMA03.tex
\def\xd{\dot{x}}%
\def\yd{\dot{y}}%
From the parametric functional,
\[
	\Phi =\lr{\xd^2 + 2\gamma\xd\yd + \yd^2}^\half - \lambda\lr{x\yd - \xd y},\quad \lambda > 0,
\]
\[
	\pderiv{\Phi}{\xd} = \half\lr{\xd^2 + 2\gamma\xd\yd + \yd^2}^{-\half}\lr{2\xd + 2\gamma\yd} + \lambda y.
\]
\[
	\pderiv{\Phi}{x} = -\lambda\yd.
\]
The \el is given by
\[
	\ELEd{\Phi}{t}{x}.
\]
Substituting into the \el for the derivatives obtained above gives,
\[
	\deriv{}{t}\lr{\half\lr{\xd^2 + 2\gamma\xd\yd + \yd^2}^{-\half}\lr{2\xd + 2\gamma\yd} + \lambda y} + \lambda\yd = 0,
\]
\[
	\deriv{}{t}\lr{\lr{\xd^2 + 2\gamma\xd\yd + \yd^2}^{-\half}\lr{\xd + \gamma\yd} + \lambda y} + \lambda\yd = 0,
\]
and integrating the above express with respect to $t$ gives,
\[
	\lrs{\lr{\xd^2 + 2\gamma\xd\yd + \yd^2}^{-\half}\lr{\xd + \gamma\yd} + \lambda y} + \lambda y = 2c,
\]
where $c$ is an arbitrary constant. So,
\begin{equation}
	\label{eq:1.1}
	\lr{\xd^2 + 2\gamma\xd\yd + \yd^2}^{-\half}\lr{\xd + \gamma\yd} = 2\lr{c - \lambda y}.
\end{equation}
Now from the definition of $s$
\begin{equation*}
	\deriv{s}{t} = \lr{\xd^2 + 2\gamma\xd\yd + \yd^2}^\half,
\end{equation*}
and
\begin{equation}
	\label{eq:1.2}
	\deriv{t}{s} = \left( 1 \middle/ \deriv{s}{t} \right) = \lr{\xd^2 + 2\gamma\xd\yd + \yd^2}^{-\half}.
\end{equation}
Substituting \eqref{eq:1.2} into \eqref{eq:1.1} gives,
\begin{equation*}
	\deriv{t}{s}\lr{\xd + \gamma\yd} = 2\lr{c - \lambda y},
\end{equation*}
and noting that
\begin{equation*}
	\deriv{x}{t} = \xd \quad \textrm{and} \quad \deriv{y}{t} = \yd,
\end{equation*}
then
\begin{align*}
	{}&\deriv{t}{s}\lr{\deriv{x}{t} + \gamma\deriv{y}{t}} = 2\lr{c - \lambda y}\\
	{}&\deriv{t}{s}\cdot\deriv{x}{t} + \gamma\deriv{t}{s}\cdot\deriv{y}{t} = 2\lr{c - \lambda y},
\end{align*}
thus, by the chain rule,
\begin{equation}
	\label{eq:1.3}
	\boxed{\deriv{x}{s} + \gamma\deriv{y}{s} = 2\lr{c - \lambda y}.}
\end{equation}
Similarly for $\yd$
\[
	\Phi =\lr{\xd^2 + 2\gamma\xd\yd + \yd^2}^\half - \lambda\lr{x\yd - \xd y},\quad \lambda > 0,
\]
\[
	\pderiv{\Phi}{\yd} = \half\lr{\xd^2 + 2\gamma\xd\yd + \yd^2}^{-\half}\lr{2\gamma\xd + 2\yd} - \lambda x.
\]
\[
	\pderiv{\Phi}{y} = \lambda\xd.
\]
The \el is given by
\[
	\ELEd{\Phi}{t}{y}.
\]
Substituting into the \el for the derivatives obtained above gives,
\[
	\deriv{}{t}\lrs{\half\lr{\xd^2 + 2\gamma\xd\yd + \yd^2}^{-\half}\lr{2\gamma\xd + 2\yd} - \lambda x} - \lambda \xd = 0,
\]
\[
	\deriv{}{t}\lrs{\lr{\xd^2 + 2\gamma\xd\yd + \yd^2}^{-\half}\lr{\gamma\xd + \yd} - \lambda x} - \lambda \xd = 0,
\]
and integrating the above express with respect to $t$ gives,
\[
	\lr{\xd^2 + 2\gamma\xd\yd + \yd^2}^{-\half}\lr{\gamma\xd + \yd} - \lambda x - \lambda x = 2d,
\]
where $d$ is an arbitrary constant. So,
\begin{equation*}
	\lr{\xd^2 + 2\gamma\xd\yd + \yd^2}^{-\half}\lr{\gamma\xd + \yd} = 2\lr{d + \lambda x}.
\end{equation*}
Recall, \eqref{eq:1.2} from above,
\begin{equation*}
	\deriv{t}{s} = \lr{\xd^2 + 2\gamma\xd\yd + \yd^2}^{-\half},
\end{equation*}
and hence,
\begin{align*}
	\deriv{t}{s}\lr{\gamma\xd + \yd} = 2\lr{d + \lambda x},\\
	\deriv{t}{s}\lr{\gamma\deriv{x}{t} +\deriv{y}{t}} = 2\lr{d+\lambda x},\\
	\gamma\deriv{t}{s}\cdot\deriv{x}{t} + \deriv{t}{s}\cdot\deriv{y}{t} = 2\lr{d+\lambda x},\\
\end{align*}
Thus, by the chain rule,
\begin{equation}
	\label{eq:1.4}
	\boxed{\gamma\deriv{x}{s} + \deriv{y}{s} = 2\lr{d + \lambda x}.}
\end{equation}
Hence, the stationary paths of the given parametric functional are given by the solutions to equations \eqref{eq:1.3} and \eqref{eq:1.4}.