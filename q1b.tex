% !TeX root = ./TMA03.tex
It can be shown that
\begin{equation*}
\lr{\deriv{x}{s}}^2 + 2\gamma\deriv{x}{s}\cdot\deriv{y}{s} + \lr{\deriv{y}{s}}^2 = 1,
\end{equation*}
as follows.

From part~(a)
\begin{equation*}
	\deriv{s}{t} = \lr{\xd^2 + 2\gamma\xd\yd + \yd^2}^\half,
\end{equation*}
and squaring both sides of the above expression gives,
\begin{equation}
	\label{eq:1.6}
	\deriv{s}{t} \cdot \deriv{s}{t}= \lr{\xd^2 + 2\gamma\xd\yd + \yd^2},
\end{equation}
Again, recall that
\begin{equation*}
	\deriv{x}{t} = \xd \quad \textrm{and} \quad \deriv{y}{t} = \yd,
\end{equation*}
so that \eqref{eq:1.6} becomes
\begin{equation}
	\label{eq:1.7}
	\deriv{s}{t} \cdot \deriv{s}{t}= \lr{\deriv{x}{t}\cdot\deriv{x}{t} + 2\gamma\deriv{x}{t}\cdot\deriv{y}{t} + \deriv{y}{t}\cdot\deriv{y}{t}},
\end{equation}
multiplying both sides of equation \eqref{eq:1.7} by
\[
	\deriv{t}{s}\cdot\deriv{t}{s}
\]
gives,
\begin{equation}
	\label{eq:1.8}
	\deriv{t}{s}\cdot\deriv{t}{s}\cdot\deriv{s}{t} \cdot \deriv{s}{t}= \deriv{t}{s}\cdot\deriv{t}{s}\cdot\deriv{x}{t}\cdot\deriv{x}{t} + 2\gamma\deriv{t}{s}\cdot\deriv{t}{s}\cdot\deriv{x}{t}\cdot\deriv{y}{t} + \deriv{t}{s}\cdot\deriv{t}{s}\cdot\deriv{y}{t}\cdot\deriv{y}{t},
\end{equation}
then by the chain rule \eqref{eq:1.8} becomes
\begin{equation}
\label{eq:1.9}
	1 = \lr{\deriv{x}{s}}^2 + 2\gamma\deriv{x}{s}\cdot\deriv{y}{s} + \lr{\deriv{y}{s}}^2,
\end{equation}
as required.
